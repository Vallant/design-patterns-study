\chapter{Abstract}

While design patterns are proposed as a standard way to achieve good software design little research is done on the actual impact of using these strategies on the code quality. Many books suggest that such methods increase flexibility and maintainability however they often lack any evidence. 

This bachelor thesis intends to empirically demonstrate that the use of design patterns actually improves code quality. 

To gather data about the code two applications were implemented, that are designed to meet the same requirements. While one application is developed following widespread guidelines and principles proposed by the object oriented programming, the other is implemented without paying attention to the topics of software maintenance. After complying to the basic requirements a number of additional features were implemented in two phases. At first a new graphical user interface is being supported, then a different data tier is added.

The results show that the initial effort of implementing the program version following object oriented programming guidelines are noticeably higher in terms of code lines and necessary files. However, during the implementation of additional features fewer files needed to be modified and during one phase transition considerably less code was needed to be written while not performing worse in the other and furthermore the cyclomatic complexity of the code increased less rapid.
