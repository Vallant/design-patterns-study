\chapter{Abstract}

While design patterns are proposed as a standard way to achieve good software design little research is done on the actual impact of using these strategies on code quality. Many books suggest that their methods increase flexibility and maintainability however they often do not provide any evidence. 

This bachelor thesis intends to empirically prove the hypothesis that the use of design patterns improves code quality. 

To gather data about the code two applications are implemented which need to meet the same requirements. While one application is developed following widespread guidelines and principles proposed by the object oriented programming, the other is implemented without paying attention to the topics of software maintenance. After complying to the basic requirements new features are implemented in two additional phases. At first a new graphical user interface is being supported, then another data tier is added.

The results show that the initial effort of implementing the program version following object oriented programming guidelines are significantly higher in terms of code lines and necessary files. However during the implementation of additional features less files needed to be modified and during one phase traversal significantly less code was needed to be written while not performing worse in the other. 
